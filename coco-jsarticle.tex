% Intended LaTeX compiler: pdflatex
\documentclass[dvipdfmx,a4j,14pt,uplatex]{jsarticle}
       \usepackage{coco-jsarticle}
\ConfidentialLevel{部外秘}
\author{島野 善雄\thanks{shimano.yoshio@jp.fujitsu.com}}
\date{\today}
\title{jsarticle 用カスタムパッケージ}
\hypersetup{
   bookmarks=true,
   bookmarksnumbered=true,
   colorlinks=true,
   setpagesize=false,
   linkcolor=blue,
   citecolor=blue,
   backref,
   pdfauthor={島野 善雄},
   pdftitle={jsarticle 用カスタムパッケージ},
   pdfkeywords={Linux \LaTeX{}},
   pdfsubject={\LaTeX{} Tips},
   pdfcreator={Emacs 25.2.2 (Org mode 9.2.2)}, 
   pdflang={Ja}}
  \begin{document}

\maketitle
\setcounter{tocdepth}{4}
\tableofcontents

\color{Black!95!White}
\section{このパッケージについて}
\label{sec:org444e444}
\index{LaTeX}

\subsection{動機}
\label{sec:orgc644d84}

\subsection{確認環境}
\label{sec:org3256ea5}
\begin{description}
\item[{OS}] Ubuntu 18.10
\item \TeX{} 環境: Texlive 2018
\end{description}

\subsection{謝辞}
\label{sec:org7582374}
このパッケージを作るにあたり、インターネット上の
たくさんの方々の情報を活用させていただきました。
深く感謝いたします。

\section{パッケージの実装}
\label{sec:orgbed9955}
このセクションでは実際のパッケージの実装を説明します。

\subsection{パッケージ宣言}
\label{sec:org1a71b2a}
まず、パッケージを宣言します。

\begin{programlist}[label={org293cccc}]{latex}{: パッケージ宣言}\NeedsTeXFormat{LaTeX2e}
\ProvidesPackage{coco-jsarticle}
   [2019/02/29 v1.0 Yoshio Shimano]
\def\FileVerjou#1{\gdef\@FileVerjou{#1}}
\end{programlist}

各行の意味は次のとおりです:

\begin{description}
\item[{\texttt{\textbackslash{}NeedsTeXFormat\{LaTeX2e\}}}] 使用するバージョンを指定します。
\item[{\texttt{\textbackslash{}ProvidesPackage\{coco-jsarticle\}}}] 提供するパッケージ名を定義します。
\item[{\texttt{[2019/02/29 v1.0 Yoshio Shimano]}}] このパッケージのバージョン情報です。
\texttt{2019/02/29} はこの書式でなければなりません。
\end{description}

\subsection{図関係の設定}
\label{sec:orga93c5e8}
\subsubsection{図をとりこむ}
\label{sec:org1dd340b}
図を取り込むために、定番の \texttt{graphicx} パッケージを使います

\begin{programlist}[label={org2dc6ae6}]{latex}{: graphicx パッケージを使う}\usepackage{graphicx}
\end{programlist}

\subsubsection{fcolorbox の説明}
\label{sec:org692ddfc}
\begin{programlist}[label={nil}]{latex}{: }\fboxsep=0.5em
\fboxrule=0pt
\end{programlist}

\subsubsection{色を使う}
\label{sec:orgf027b93}
色を使うために、 \texttt{xcolor} を使っています。

\begin{programlist}[label={org9310bf0}]{latex}{: xcolor を使う}\usepackage[usenames,divpsnames,svgnames,table,hyperref]{xcolor}
\end{programlist}

\subsubsection{色を定義する}
\label{sec:org2838d54}
Bootstrap の色を定義します。
Bootstrap で使われている色は次のとおりです:

\begin{table}[htbp]
\caption{\label{tab:org3278755}
bootstrap の色定義}
\centering
\begin{tabular}{rll}
\hline
 & クラス名 & 色コード\\
\hline
1 & text-primary & \#007BFF\\
2 & text-secondary & \#6C757D\\
3 & text-success & \#28A745\\
4 & text-info & \#17A2B8\\
5 & text-warning & \#FFC107\\
6 & text-danger & \#DC3545\\
7 & text-dark & \#343A40\\
8 & text-muted & \#6C757D\\
9 & text-light & \#F8F9FA\\
10 & text-white & \#FFFFFF\\
\hline
\end{tabular}
\end{table}

\begin{programlist}[label={nil}]{latex}{: }% 色の定義
\definecolor{text-primary}{HTML}{007BFF}
\definecolor{text-secondary}{HTML}{6C757D}
\definecolor{text-success}{HTML}{28A745}
\definecolor{text-info}{HTML}{17A2B8}
\definecolor{text-warning}{HTML}{FFC107} 
\definecolor{text-danger}{HTML}{DC3545}
\definecolor{text-dark}{HTML}{343A40}
\definecolor{text-muted}{HTML}{6C757D}
\definecolor{text-light}{HTML}{F8F9FA}
\definecolor{text-white}{HTML}{FFFFFF}

\definecolor{teal}{RGB}{0,128,128}
\definecolor{powderblue}{RGB}{176,224,230}
\definecolor{darkslateblue}{RGB}{72,61,139}
\definecolor{darkslategray}{RGB}{47,79,79}
\definecolor{lightcyan}{RGB}{224,255,255}
\end{programlist}


\begin{itemize}
\item text-primary --- \colorbox{text-primary}{text-primary}
\item text-secondary--- \colorbox{text-secondary}{text-secondary}
\item text-success--- \colorbox{text-success}{text-success}
\item text-info--- \colorbox{text-info}{text-info}
\item text-warning--- \colorbox{text-warning}{text-warning}
\item text-danger--- \colorbox{text-danger}{text-danger}
\item text-dark --- \colorbox{text-dark}{text-dark}
\item text-muted--- \colorbox{text-muted}{text-muted}
\item text-light--- \colorbox{text-light}{text-light}
\item text-white--- \colorbox{text-white}{text-white}
\item teal        --- \colorbox{teal}{teal}
\item powderblue  --- \colorbox{powderblue}{powderblue}
\item darkslateblue  --- \colorbox{darkslateblue}{darkslateblue}
\item darkslategray  --- \colorbox{darkslategray}{darkslategray}
\item lightcyan  --- \colorbox{lightcyan}{lightcyan}
\end{itemize}

\subsubsection{tikz 設定}
\label{sec:org51c67b2}
tikz 設定 の設定です。後述の tcolorbox のために
使っています。

\begin{programlist}[label={orgd7b5eb8}]{latex}{: tikz 設定}% tikz を使う
\usepackage{tikz}
\usetikzlibrary{shadings,shadows}
\usetikzlibrary{decorations.pathmorphing}
\usetikzlibrary{patterns}
\usetikzlibrary{spy}
\usetikzlibrary{arrows.meta}
\end{programlist}

\subsection{フォントの設定}
\label{sec:org9cd3a9e}
文書で使うフォントの設定を行ないます。

\subsubsection{欧文フォントの設定}
\label{sec:orgf25c4c8}
欧文フォントを使う設定です。

欧文フォントには、 Latin Modern を使っています。

\begin{programlist}[label={org3b3fb10}]{latex}{: 欧文フォントに必用なパッケージ}\usepackage[T1]{fontenc}
\usepackage{textcomp}
\usepackage[lmr]{mathcomp}
\usepackage[utf8]{inputenc}
\usepackage{lmodern} % Latin Modern を使う
\end{programlist}
\subsubsection{数式フォントの設定}
\label{sec:org574adfc}
『数学ガール』のファンなので、
数式用フォントには \texttt{eulervm} を使います。

\begin{programlist}[label={org29276c1}]{latex}{: 数式フォントの設定}% 数式フォント
% \usepackage{mathpazo}
\usepackage{eulervm}
% \usepackage{newpxtext,newpxmath}
\end{programlist}

\subsubsection{和文フォントの設定}
\label{sec:orgdef2485}
和文フォントは OTF パッケージを使っています。

こんな文字が出したいです。

\begin{itemize}
\item 白鴎と白鷗
\item 「吉野家」と「𠮷野家」
\item 森鷗外と内田百閒とが髙島屋に行くところを想像した。
\item 葛飾区の𠮷野家
\item Macintosh用キーボードの⌘(Command key)を押す。
\item ♲ を心がけよう。
\end{itemize}

\begin{programlist}[label={org694b561}]{latex}{: otf パッケージ}\usepackage[uplatex,jis2004,expert,deluxe]{otf}
\end{programlist}

\begin{description}
\item[{uplatex}] upLaTeX を使います。
\item[{jis2004}] 可能であれば、jis2004 の文字を使います。
\item[{expert }] 横書きと縦書きで違う文字を使います。
\item[{deluxe }] 書体がたくさん使えるようになります。
\end{description}

\subsection{背景色の変更}
\label{sec:orgc89b860}
ページの背景色をちょっとだけ白くします。

\begin{programlist}[label={nil}]{latex}{: }\usepackage[pagecolor={White!95!Black}]{pagecolor}
\end{programlist}

\subsection{テキストの調整}
\label{sec:org42231d1}
\subsubsection{テキストをページ一杯にひろげる}
\label{sec:orgdab77cd}
テキストをページ一杯に拡げます。
jsarticle では効果はないかもしれません。

\begin{programlist}[label={org46e0a8e}]{latex}{: テキストをページ一杯に拡げる}\setlength{\textwidth}{\fullwidth}
\end{programlist}

\subsubsection{行間の変更}
\label{sec:orge8ec60d}
行間を拡げます。

行間を調節します。
広いほうが好みなので、全角の高さ(=zh=)の 0.8 倍にしています。

\begin{programlist}[label={orgb35ce31}]{latex}{: 行間の設定}% 行間の設定
\setlength{\baselineskip}{0.8 ex}
\end{programlist}

\subsubsection{段落の調整}
\label{sec:orga1e583f}
段落の先頭にあるインデントをなくし、
段落間の空きをふやします。

\begin{itemize}
\item \href{https://tex.stackexchange.com/questions/358588/parskip-and-title-spacing-conflict}{titlesec - parskip and title spacing conflict - \TeX{} - \LaTeX{} Stack Exchange}
\end{itemize}

を参考にしました。

\begin{itemize}
\item \href{https://github.com/FrankMittelbach/fmitex/tree/master/parskip}{fmitex/parskip at master · FrankMittelbach/fmitex · GitHub}
\end{itemize}

から、 \texttt{parskip} パッケージをいただいてきてください。
次のようにすると、段落間にスペースがあきます。


\begin{programlist}[label={orgb4fd86c}]{latex}{: 段落のインデントをなくし、段落の間を空ける}% 段落のインデントをなくし、段落の間を空ける
\usepackage[skip=1.2em]{parskip}
\end{programlist}

\subsubsection{uline-- を使っていろいろな下線をひく}
\label{sec:org8b065a0}
\index{uline--.sty}

\begin{itemize}
\item \href{http://doratex.hatenablog.jp/entry/20171219/1513609345}{行分割可能な \texttt{\textbackslash{}fbox} をつくる - \TeX{} Alchemist Online}
\end{itemize}

を参考にしました。

\texttt{uline-{}-.sty} を使います。
標準パッケージではありません。

\begin{itemize}
\item \href{https://github.com/doraTeX/breakfbox}{GitHub - doraTeX/breakfbox}
\end{itemize}

からダウンロードしてください。

\begin{programlist}[label={org9b7d01c}]{latex}{: \texttt{uline-{}-.sty} を使う\texttt{uline-{}-.sty} を使う}\usepackage[usetype1]{uline--}
\end{programlist}

このパッケージを使うと次のようなことができます。

\begin{programlist}[label={org88e5984}]{latex}{: \texttt{uline-{}-.sty} の使用例}\uline{下線}、\mline{打消線}、\oline{上線}、
\udash{下破線}、\mdash{打消破線}、\odash{上破線}
\uwave{下波線}、\mwave{打消波線}、\owave{上波線}
\uline[background,color={[rgb]{1,1,0.4}},width=0.5zw,position=1pt]{蛍光ペン}
\uline[background,color={[rgb]{1,0.4,1}},width=1zw,position=.38zw]{塗り}、
\end{programlist}

\uline{下線}、\mline{打消線}、\oline{上線}、
\udash{下破線}、\mdash{打消破線}、\odash{上破線}
\uwave{下波線}、\mwave{打消波線}、\owave{上波線}
\uline[background,color={[rgb]{1,1,0.4}},width=0.5zw,position=1pt]{蛍光ペン}
\uline[background,color={[rgb]{1,0.4,1}},width=1zw,position=.38zw]{塗り}、

*

\subsubsection{文字の強調の変更}
\label{sec:orga2f14bc}
最近の HTML でみかけるように、強調の文が
蛍光マーカーで線を引かれたようにします。
\LaTeX の \texttt{\textbackslash{}emph} コマンドを再定義します。

\begin{description}
\item[{日本語}] ゴシック太字
\item[{欧文}] イタリック太字
\item[{塗り}] 黄色で、文字の半分まで
\end{description}

というようになるように、 \texttt{emph} コマンドを変更します。

\begin{programlist}[label={orge02e1b9}]{latex}{: \texttt{emph} コマンドの再定義}\usepackage[usetype1]{uline--}

\renewcommand{\emph}[1]{%
  {\sffamily\bfseries\itshape%
    \uline[
      background,
      color={[rgb]{1,1,0.0}},
      width=0.8em,position=1pt]{#1}}}
\end{programlist}

\emph{強調のテキストです。 This is an emph.}

これを使うと、これは:
\begin{programlist}[label={orgebe1bd7}]{text}{: 強調の例}/強調の行です。 This is emph/ 。うまくいくかな?
\end{programlist}

このように変換されます。

\emph{強調の行です。 This is emph} 。うまくいくかな?

\subsubsection{打ち消し線の定義}
\label{sec:orga0b1a4c}
\sout{\sout{Strike through}} の文字を出します。
Org mode が打ち消し線に対して \texttt{sout} を
使うので、 \texttt{sout} コマンドを定義します。

\begin{programlist}[label={org013f459}]{latex}{: 打ち消し線の定義}\newcommand{\sout}[1]{\mline{#1}}
\end{programlist}



\subsection{レイアウトの変更}
\label{sec:orgc07b4c4}
\begin{description}
\item[{左側余白}] 10mm
\item[{右側余白}] 10mm
\item[{上側余白}] 25mm
\item[{下側余白}] 25mm
\end{description}

になるように、余白を設定します


\begin{programlist}[label={org5725cd3}]{latex}{: 余白の設定}\usepackage[top=25truemm,bottom=25truemm,inner=10truemm,outer=10truemm]{geometry}
\end{programlist}
\subsection{目次の変更}
\label{sec:orga279aa8}
目次の見栄えを変更します。

\begin{itemize}
\item \href{https://github.com/thortex/jlatex-man-lll-jou}{thortex/jlatex-man-lll-jou: Japanese \LaTeX{} manual: "Love Love \LaTeX{} for Beginners"}
\end{itemize}


を参考にしました。

\subsubsection{主目次の変更}
\label{sec:org9a505d3}
主目次の見栄えを変更します。

\begin{programlist}[label={org0918a33}]{latex}{: 主目次の変更}\renewcommand{\tableofcontents}{%
  \if@twocolumn
  \else
    \@restonecolfalse
  \fi
  \section*{\contentsname%
 	\@mkboth{\contentsname}{\contentsname}%
  	\pdfbookmark{\contentsname}{contents}}
  \@starttoc{toc}%
  \if@restonecol\twocolumn\fi
}

% 色の設定
\def \@default@gray@level {.15}

% 目次のセクションレベルの変更
\renewcommand{\l@section}[2]{%
  \ifnum \c@tocdepth >\m@ne
    \addpenalty{-\@highpenalty}%
    \addvspace{.5\cvs \@plus \p@ \@minus \p@}
    \begingroup
      \parindent = \z@ \relax
      \rightskip = \@tocrmarg \relax
      \parfillskip = -\rightskip \relax
      \leavevmode \normalsize \sffamily
      \@lnumwidth = 4.683em\relax
      \advance \leftskip \@lnumwidth \hskip-\leftskip
      \hb@xt@ \z@{\color[cmyk]{0,0,0,\@default@gray@level}%
          \vrule \@height 1em \@width 3pt \@depth 1ex\hss}%
      \hskip 6pt #1\nobreak\hfill\nobreak\hb@xt@\@pnumwidth{\hss#2}\par
        {\color[cmyk]{0,0,0,\@default@gray@level}%
          \hrule \@width \linewidth \@height 0.5ex}%
      \par\nobreak\vskip6pt
      \penalty\@highpenalty
    \endgroup
  \fi
}

\renewcommand*{\l@subsection}{\@dottedtocline{1}{1em}{3em}}
\renewcommand*{\l@subsubsection}{\@dottedtocline{2}{3em}{4em}}

% 行の終わりまでドットを描く
\def\@dottedtocline#1#2#3#4#5{\ifnum #1>\c@tocdepth \else
  \vskip \z@ \@plus.2\p@
  {%\ifnum#1=2\small\fi
    \leftskip #2\relax \rightskip \@tocrmarg \parfillskip -\rightskip
    \parindent #2\relax\@afterindenttrue
   \interlinepenalty\@M
   \leavevmode
   \@lnumwidth #3\relax
   \advance\leftskip \@lnumwidth \null\nobreak\hskip -\leftskip
    {#4}\nobreak
    \leaders\hbox{$\m@th \mkern \@dotsep mu\hbox{.}\mkern \@dotsep
       mu$}\hfill \nobreak\hb@xt@\@pnumwidth{%
         \hfil%\ifnum#1=2\normalsize\fi
         \normalfont \normalcolor #5}\par}\fi}

\end{programlist}
\subsubsection{図目次の変更}
\label{sec:org66c4181}
図目次を変更します。

\begin{programlist}[label={org5e88472}]{latex}{: 図目次の変更}% 図目次の変更
\renewcommand{\listoffigures}{%
  \if@twocolumn\@restonecoltrue\onecolumn
  \else\@restonecolfalse\fi
  \section*{\listfigurename % \section* レベル
    \@mkboth{\listfigurename}{\listfigurename}%
      \pdfbookmark{\listfigurename}{listoffigures}}% append
  \@starttoc{lof}%
  \if@restonecol\twocolumn\fi
}
\end{programlist}

\subsubsection{表目次の変更}
\label{sec:orgaefbaec}
表目次を変更します。

\begin{programlist}[label={orgf57b8d7}]{latex}{: 表目次の変更}% 表目次の変更
\renewcommand{\listoftables}{%
  \if@twocolumn\@restonecoltrue\onecolumn
  \else\@restonecolfalse\fi
  \section*{\listtablename % \section* レベル
  \@mkboth{\listtablename}{\listtablename}%
  \pdfbookmark{\listtablename}{listoftables}}% append
  \@starttoc{lot}%
  \if@restonecol\twocolumn\fi
}
\end{programlist}

\subsection{ページのヘッダーとフッターの変更}
\label{sec:org34d54de}
ページのヘッダーとフッターを変更します。

\tcbox[colback=white,colframe=red,size=small,on line]{
  \textcolor{red}{\sffamily \bfseries 部外秘}
  }


まず、「部外秘」などを入れる変数 \texttt{ConfidentialLevel} を定義します。


\begin{programlist}[label={org5cf4084}]{latex}{: ConfidentialLevel の定義}\global\let\@ConfidentialLevel\@empty
\def\ConfidentialLevel#1{\gdef\@ConfidentialLevel{#1}}
\end{programlist}

プリアンブルの中で、次のように定義してください。

\begin{exampleoutput}
\ConfidentialLevel{部外秘}
\end{exampleoutput}

定義されていなければ出力されません。

次に \texttt{fancyhdr} パッケージを使って、ヘッダーとフッターを変更します。

\begin{programlist}[label={orgd917064}]{latex}{: ヘッダーとフッターの変更}\usepackage{fancyhdr}
\pagestyle{fancy}
% \lhead{\rightmark}
\rfoot{% フッター右側に「部外秘」を出力
  \ifx\@ConfidentialLevel\@empty
  \else
    \tcbox[colback=white,colframe=red,size=small,on line]{
      \textcolor{red}{\sffamily \bfseries {\@ConfidentialLevel}}
  }\fi%
}
\cfoot{\thepage}% フッター中央にページ番号を出力
\end{programlist}

\tcbset{colback=white,colframe=red}
\foreach \s in {normal,title,small,fbox,tight,minimal} {
\tcbox[size=\s,on line]{\s} }
\tcbox[colback=white,colframe=red,size=small,on line]{
  \textcolor{red}{\sffamily \bfseries 関係者外秘}
  }

\subsection{tcolorbox を使った綺麗な箱}
\label{sec:org3322180}
tcolorbox パッケージを使うと、
箱にはいった環境を比較的簡単に作ることができます

ドキュメントを見るには次のコマンドを使用してください:

\begin{exampleoutput}
texdoc tcolorbox
\end{exampleoutput}

\subsubsection{tcolorbox の設定}
\label{sec:org4722690}
tcolorbox の設定です。

\begin{programlist}[label={orgcf70e00}]{latex}{: tcolorbox 設定fcolorbox の設定}\usepackage{tcolorbox}
\tcbuselibrary{breakable,skins,raster,listings}
\tcbuselibrary{external}
\tcbuselibrary{minted}
\tcbEXTERNALIZE
\end{programlist}

\subsubsection{pabox 環境}
\label{sec:org46a5298}
タイトル, ラベル(オプション) つきのボックスです。

\begin{programlist}[label={orgaa77748}]{latex}{: タイトル, ラベル(オプション) つきのボックス}\newtcolorbox[
  auto counter,
  number within=section]{pabox}[2][]{%
  colback=red!5!white,
  colframe=red!75!black,
  fonttitle=\bfseries,
  title=例~\thetcbcounter: #2,#1}
\end{programlist}

\subsubsection{marker 環境}
\label{sec:orgbea3274}
\begin{programlist}[label={org8578a48}]{latex}{: maraker 環境}\newtcolorbox{marker}[1][]{enhanced,
  before skip=2mm,
  after skip=3mm,
  boxrule=0.4pt,
  left=5mm,
  right=2mm,
  top=1mm,
  bottom=1mm,
  colback=yellow!50,
  colframe=yellow!20!black,
  sharp corners,
  rounded corners=southeast,
  arc is angular,
  arc=3mm,
  underlay={%
    \path[fill=tcbcol@back!80!black] ([yshift=3mm]interior.south east)--++(-0.4,-0.1)--++(0.1,-0.2);
    \path[draw=tcbcol@frame,shorten <=-0.05mm,shorten >=-0.05mm] ([yshift=3mm]interior.south east)--++(-0.4,-0.1)--++(0.1,-0.2);
    \path[fill=yellow!50!black,draw=none] (interior.south west) rectangle node[white]{\Huge\bfseries !} ([xshift=4mm]interior.north west);
    },
  drop fuzzy shadow,#1}
\end{programlist}
\subsubsection{programlist 環境}
\label{sec:orgdf339ea}
\index{programlist}
\begin{programlist}[label={org7cdb521}]{latex}{: プログラムリスト用環境 \texttt{programlist}}\newtcblisting[
  auto counter,
  number within=section,
  list inside=box]{programlist}[3][]{
  listing engine=minted,% リスト環境は minted
  minted style=monokai,% 使用するテーマ
  minted language=#2, % 使用する言語
  minted options={fontsize=\small,
                  breaklines,% 途中で改行する
                  breakanywhere},%
  title={\sffamily\bfseries リスト \thetcbcounter #3},
  #1,% ラベル
  breakable,%
  colback=black!90!white,
  colupper=white,
  colframe=blue!75!white,
  listing only,%
  left=0mm,
  enhanced,%
   }%
\end{programlist}

このようにして使います:

\begin{exampleoutput}
\begin{programlist}[label={prog-exam1}]{lisp}{: Emacs Lisp の例}(defun org-xor (a b)
  "Exclusive or."
  (if a (not b) b))
\end{programlist}
\end{exampleoutput}

\begin{programlist}[label={prog-exam1}]{lisp}{: Emacs Lisp の例}(defun org-xor (a b)
  "Exclusive or."
  (if a (not b) b))
\end{programlist}

\subsubsection{shellinput 環境}
\label{sec:org9a5a3d2}
シェル入力用環境の環境です。

\begin{programlist}[label={org3dab0cf}]{latex}{: シェル入力用環境}\newtcblisting{shellinput}[1][]{
  colback=black,
  colupper=white,
  colframe=blue!75!white,
  listing engine=minted,
  title=#1,
  listing only,
  minted language=shell-session,
  minted options={fontsize=\footnotesize},
  breakable,
  minted style=monokai
}
\end{programlist}

\subsubsection{shelloutput 環境}
\label{sec:org9b9ef65}
シェル出力用環境です。

\begin{programlist}[label={org8ba6bb6}]{latex}{: shelloutput 環境}\newtcblisting{shelloutput}[1][]{
  colback=black,
  colupper=white,
  colframe=blue!75!white,
  listing engine=minted,
  title=#1,
  listing only,
  minted language=shell-session,
  minted options={fontsize=\footnotesize},
  breakable,
  minted style=monokai
}
\end{programlist}

\subsubsection{例の出力}
\label{sec:org6f817b2}
\begin{programlist}[label={org46a7f19}]{latex}{: exampleoutput 環境}\newtcblisting{exampleoutput}{
  colback=black,
  colupper=white,
  colframe=blue!75!white,
  listing engine=minted,
%  title=出力,
  listing only,
  minted language=text,
  minted options={fontsize=\footnotesize},
  breakable,
  minted style=monokai
}
\end{programlist}

\subsubsection{リスト目次関係}
\label{sec:orgce8e624}
\begin{programlist}[label={org6964208}]{latex}{: リスト目次の先頭に番号を出さない}\makeatletter
\def\tcb@addcontentsline#1#2{%
  \ifx\kvtcb@listentry\@empty%
    \ifx\kvtcb@title\@empty%
      \ifx\tcbtitletext\@empty%
        \addcontentsline{#1}{#2}{{\ignorespaces\kvtcb@savedelimiter}}%
      \else%
        \addcontentsline{#1}{#2}{{\ignorespaces\tcbtitletext}}%
      \fi%
    \else%
      \addcontentsline{#1}{#2}{{\ignorespaces\kvtcb@title}}%
    \fi%
  \else%
    \addcontentsline{#1}{#2}{\kvtcb@listentry}%
  \fi%
}
\makeatother
\end{programlist}

\subsubsection{プログラムリストのキャプションの変更}
\label{sec:org7bd505c}
\begin{programlist}[label={org8212663}]{latex}{: プログラムリストのキャプション}\renewcommand\listingscaption{プログラムコード}
\end{programlist}

\subsubsection{プログラムリストの目次のキャプションの変更}
\label{sec:org2ebe5c4}
\begin{programlist}[label={org01f37bd}]{latex}{: プログラムリストの目次のキャプション}\renewcommand\listoflistingscaption{プログラムコードのリスト}
\end{programlist}

\begin{programlist}[label={nil}]{latex}{: }\tcbset{colback=white,colframe=red}
\end{programlist}

\subsection{セクションの見出しの変更}
\label{sec:org7916ecf}
デフォルトの jsclass の設定では見出しが
あじけないので、変更します。

fancydr の後で titlesec を読みこまないと、
うまくいかないようです。

\subsubsection{titlesec を使う}
\label{sec:org6461744}
見出しのみかけを変更するために \texttt{titlesec} パッケージを
使います。

\begin{programlist}[label={org334ec72}]{latex}{: titlesec パッケージの読み込み}\usepackage{titlesec}
% \usepackage{anyfontsize}
\end{programlist}
\subsubsection{titlesec の設定方法}
\label{sec:orgeaaf4dd}
\begin{itemize}
\item \href{http://abenori.blogspot.com/2018/05/titlesec.html}{にっき♪: titlesec}
\end{itemize}

を参考にしました。

\texttt{\textbackslash{}titleformat} コマンドの書式です:

\begin{programlist}[label={orgd705a0c}]{latex}{: \texttt{\textbackslash{}titleformat} の書式}\titleformat{<命令>}
    [<特殊な形状の指定>]
    {<書式>}
    {<ラベル書式>}
    {<ラベルと見出し文字列の間の空き>}
    {<見出し文字列直前に入るコード>}
    [<見出し直後に入るコード>]
\end{programlist}

\subsubsection{セクションの前で改ページを行なう}
\label{sec:orgbea2746}
セクションの前で改ページを行なう設定です。

セクションの直前で改ページを行なうには
次のようにします。

\begin{programlist}[label={orgeda047d}]{latex}{: セクションの直前で改ページを行なう}% セクションの前で改ページを行なう
\newcommand{\sectionbreak}{\clearpage}
\end{programlist}

\subsubsection{section の見出しの変更}
\label{sec:orgbaa97d2}
セクションのスタイルを変更します。

\begin{programlist}[label={orgdf9e9b8}]{latex}{: section のスタイルの変更}\titleformat{\section}
[hang]
{\Large\sffamily\bfseries}
{\colorbox{blue}{\color{white}\thesection}}{12pt}{}%
[{\titlerule[0.5pt]}]
\end{programlist}

\subsubsection{section* の見出しの変更}
\label{sec:org8af10e7}
\begin{programlist}[label={org8864757}]{latex}{: section* のスタイルの変更}\titleformat{name=\section,numberless}
[hang]
{\Large\sffamily\bfseries}
{}{12pt}{}%
[{\titlerule[0.5pt]}]
\end{programlist}

\subsubsection{subsection のスタイルの変更}
\label{sec:orge83ff0a}
サブセクションのスタイルを変更します。

\begin{programlist}[label={org82609e6}]{latex}{: subsection のスタイルの変更}\titleformat{\subsection}
[hang]
{\large\sffamily\bfseries}
{\colorbox{teal}{\color{white}\thesubsection}}{12pt}{}%
[{\titlerule[0.5pt]}]
\end{programlist}

\subsubsection{subsubsection のスタイルの変更}
\label{sec:org11a4628}
subsubsection スタイルを変更します。

\subsubsection{paragraph のスタイルの変更}
\label{sec:org5e8c6fa}
paragraph のスタイルを変更します。

\begin{programlist}[label={org39a9f85}]{latex}{: subsubsection のスタイルの変更paragraph のスタイルの変更}\titleformat{\subsubsection}
[hang]
{\large\sffamily\bfseries}
{\colorbox{darkslateblue}{\color{white}\thesubsubsection}}{12pt}{}%
[{\titlerule[0.5pt]}]
\end{programlist}



\subsubsection{セクションの空きを設定する}
\label{sec:org04de5d3}
セクションの見出しの空きを設定します。

書式は次のとおりです:

\begin{programlist}[label={org8a2890c}]{latex}{: \texttt{titlespacing} の書式}\titlespacing*{\section}{左空き}{上空き}{下空き}
\end{programlist}

次のように設定しました。好みに合わせて
変更してください。

\begin{programlist}[label={orgb696661}]{latex}{: titlespacing によって見出しの空きを設定する}\titlespacing*{\section}{0em}{2em}{2em}
\titlespacing*{\subsection}{0em}{2em}{2em}
\titlespacing*{\subsubsection}{0em}{2em}{2em}
\titlespacing*{\paragraph}{0em}{2em}{2em}
\end{programlist}


\subsection{表にストライプをつける}
\label{sec:org54d53b7}
表にストライプをつけます。

\begin{itemize}
\item \href{https://tex.stackexchange.com/questions/61747/how-to-apply-alternate-row-coloring-in-a-longtable-in-lyx}{color - How to apply alternate row coloring in a longtable in LyX? - \TeX{} - \LaTeX{} Stack Exchange}
\end{itemize}

を参考にしました。


\begin{programlist}[label={orgfa671a8}]{latex}{: 表にストライプをつける}% define lightgray
\definecolor{lightgray}{gray}{0.9}

% alternate rowcolors for all tables
\let\oldtabular\tabular
\let\endoldtabular\endtabular
\renewenvironment{tabular}{\rowcolors{2}{white}{lightgray}\oldtabular}{\endoldtabular}
\end{programlist}

このような表が出力できます:

\begin{table}[htbp]
\caption{表のテスト}
\centering
\begin{tabular}{ll}
コード & 表示\\
\hline
\texttt{\textbackslash{}alpha} & \(\alpha\)\\
\texttt{\textbackslash{}beta} & \(\beta\)\\
\texttt{\textbackslash{}gamma} & \(\gamma\)\\
\texttt{\textbackslash{}delta} & \(\delta\)\\
\hline
\end{tabular}
\end{table}

\subsection{箇条書きの変更}
\label{sec:org99ddfac}
標準の箇条書きの不満な点です:

\begin{itemize}
\item 箇条書きのネストが 4 つまで
\item ちょっと見た目がかっこう悪い。
\end{itemize}

箇条書きを変更するため、 \texttt{enumitem} パッケージを使用します。

\begin{programlist}[label={org7744dce}]{latex}{: \texttt{enumitem} パッケージの使用}\usepackage{enumitem}
\end{programlist}


\subsubsection{箇条書きネストのレベルの設定}
\label{sec:orgf743e70}
\LaTeX はデフォルトで 4 つのレベルの箇条書きのネストを
サポートしています。しかしそれでは足りないことがあるので、
そのネストのレベルを深くします。

\begin{programlist}[label={org8b8561c}]{latex}{: 箇条書きのネストのレベルを深くする}\setlistdepth{20}
\end{programlist}

\subsubsection{itemize の再定義}
\label{sec:org0b526f1}
itemize 環境を再定義します。

\begin{programlist}[label={org6db231a}]{latex}{: itemize の再定義}\renewlist{itemize}{itemize}{20}
\setlist[itemize]{
  label=\textbullet, 
  partopsep=0em,
  parsep=0.3em,
  labelindent=2em,
  leftmargin=2em}
\end{programlist}

例です。この Org mode の箇条書きは:

\begin{programlist}[label={org205228d}]{text}{: Org mode itemize の例}- レベル1
  - レベル2
    - レベル3
      - レベル4
        - レベル5
- レベル1
  - レベル2
    - レベル3
      - レベル4
        - レベル5
\end{programlist}

このように変換されます:

\begin{itemize}
\item レベル1
\begin{itemize}
\item レベル2
\begin{itemize}
\item レベル3
\begin{itemize}
\item レベル4
\begin{itemize}
\item レベル5
\end{itemize}
\end{itemize}
\end{itemize}
\end{itemize}
\item レベル1
\begin{itemize}
\item レベル2
\begin{itemize}
\item レベル3
\begin{itemize}
\item レベル4
\begin{itemize}
\item レベル5
\end{itemize}
\end{itemize}
\end{itemize}
\end{itemize}
\end{itemize}

\subsubsection{enumerate の再定義}
\label{sec:org0814a64}
enumerate 環境を再定義します。

\begin{programlist}[label={org6320190}]{latex}{: enumerate の再定義}\renewlist{enumerate}{enumerate}{20}
  \setlist[enumerate]{
    leftmargin=!, 
    font=\sffamily\bfseries,
    label*=\arabic*.,% ラベルを 1., 1.1. 1.1.1 に
%    label=\arabic*, 
    itemindent=!, 
    topsep=0.3em,
    partopsep=0.3em,
    parsep=0.3em,
    labelsep=!, 
    labelwidth=!, 
    labelindent=2em}
\end{programlist}

これは

\begin{programlist}[label={org90d6b07}]{text}{: Org mode での順序つきリストアイテム}1. レベル 1
   1. レベル 2
      1. レベル 3
         1. レベル 4
            1. レベル 5
2. レベル 1
   1. レベル 2
      1. レベル 3
         1. レベル 4
            1. レベル 5
\end{programlist}

このように変換されます:


\begin{enumerate}
\item レベル 1
\begin{enumerate}
\item レベル 2
\begin{enumerate}
\item レベル 3
\begin{enumerate}
\item レベル 4
\begin{enumerate}
\item レベル 5
\end{enumerate}
\end{enumerate}
\end{enumerate}
\end{enumerate}
\item レベル 1
\begin{enumerate}
\item レベル 2
\begin{enumerate}
\item レベル 3
\begin{enumerate}
\item レベル 4
\begin{enumerate}
\item レベル 5
\end{enumerate}
\end{enumerate}
\end{enumerate}
\end{enumerate}
\end{enumerate}

\subsubsection{desctiption の再定義}
\label{sec:orga04d309}
\texttt{desctiption} 環境を再定義します。

\begin{programlist}[label={orga533277}]{latex}{: desctiption の再定義}\renewlist{description}{description}{20}
\setlist[description]{
  font=\sffamily\bfseries, 
  style=nextline,
  labelindent=2em,% ラベルのインデント量
  }
\end{programlist}
これは

\begin{programlist}[label={orge69d682}]{text}{: Org mode での説明つきリスト}- 用語1 :: 用語の説明1。
- 用語2 :: 用語の説明2。長くなるとどうなるだろうか? うまくいくかな?
         ちゃんと複数行に行くかな。
- とってもとってもとっても長い用語 :: 
     用語の説明2。長くなるとどうなるだろうか? うまくいくかな?
     ちゃんと複数行に行くかな。
\end{programlist}

このように変換されます:


\begin{description}
\item[{用語1}] 用語の説明1。
\item[{用語2}] 用語の説明2。長くなるとどうなるだろうか? 
うまくいくかな?
ちゃんと複数行に行くかな。
\item[{とってもとってもとっても長い用語}] 用語の説明2。長くなるとどうなるだろうか? 
うまくいくかな?
ちゃんと複数行に行くかな。
\end{description}

\subsection{数式}
\label{sec:org1065506}
とっても一般的な数式の設定です。

\begin{programlist}[label={org0f8b937}]{latex}{: AMSMath を使う}\usepackage{amsmath,amssymb}
\usepackage{bm}
\end{programlist}

Org mode の中で次のように書くと:

\begin{programlist}[label={org4f5b2f1}]{text}{: Org mode での数式の例1}\begin{equation}
  A = \begin{pmatrix}
        a_{11} & \ldots & a_{1n} \\
        \vdots & \ddots & \vdots \\
        a_{m1} & \ldots & a_{mn}
      \end{pmatrix}
\end{equation}
\end{programlist}

このような数式が出力されます:

\begin{equation}
  A = \begin{pmatrix}
        a_{11} & \ldots & a_{1n} \\
        \vdots & \ddots & \vdots \\
        a_{m1} & \ldots & a_{mn}
      \end{pmatrix}
\end{equation}

\subsection{hyperref 設定}
\label{sec:orgac3aa77}
hyperref 関連の設定です。

\subsubsection{hyperref を使う}
\label{sec:orgf61bd89}
hyperref を使う設定です。

\begin{programlist}[label={orgbeaabea}]{latex}{: hyperref を読み込む}\usepackage{hyperref}
\end{programlist}

\subsubsection{PDF のしおりの文字化けを防ぐ}
\label{sec:org4252c5f}
\begin{programlist}[label={org8583ae4}]{latex}{: PDF の文字化けを防ぐ}% PDF の文字化けを防ぐ
\usepackage{pxjahyper}
\end{programlist}


Org mode のファイルのどこかに次の行を書いておくと、
PDF のキーワードが設定されます。

\begin{programlist}[label={org01d3b26}]{text}{: PDF 用キーワードの設定}#+KEYWORDS:  upLaTeX tcolorbox
\end{programlist}

\subsection{索引の作成}
\label{sec:org932125b}
\index{makeindex}
\index{makeidx}

索引を出力するには \texttt{makeidx} パッケージを使います。

\begin{programlist}[label={org2dca431}]{latex}{: 索引出力用のパッケージの読みこみ}\usepackage{makeidx}
\end{programlist}

そして次に、 \texttt{makeindex} コマンドを使って索引を作成します。

\begin{programlist}[label={org238a499}]{latex}{: 索引作成}\makeindex
\end{programlist}



\section{latexmk によるコンパイルの自動化}
\label{sec:org4bd0dfe}
\index{.latexmkrc}
\index{latexmk}

latexmk を使うと、面倒な \LaTeX ファイルの
コンパイルを自動化してくれます。

\subsection{\texttt{.latexmkrc} の書きかた}
\label{sec:org3453061}
\texttt{.latexmkrc} の例です。

\begin{programlist}[label={orgd2f764d}]{shell}{: \texttt{.latexmkrc} の例}# LaTeX コマンドの設定
$latex  = 'uplatex -src-specials -shell-escape -synctex=1 -interaction=nonstopmode';

# BibTeX コンパイラ
$bibtex = 'upbibtex';
# $bibtex = 'biber';

$dvipdf  = 'dvipdfmx %O -o %D %S';

$makeindex  = '/usr/bin/mendex -U -r -c -g -d ~/texmf/makeindex/dict/main.dict -s ~/texmf/makeindex/ist/dot.ist -p any';

$pdf_previewer = 'xdg-open %O %S';

# 最大の繰り返し回数
$max_repeat       = 5;

# 0:pdf化しない場合
# 1:pdflatexを使う場合
# 2:ps2pdfを使う場合
# 3:dviファイルからpdfを作成する場合
$pdf_mode = 3;

$pdf_update_method = 0;

\end{programlist}

\subsection{latexmk の使い方}
\label{sec:orgfccfbf6}
\begin{exampleoutput}
latexmk foge.tex
\end{exampleoutput}


\appendix

\section{ライセンス}
\label{sec:org0de9148}
MIT。


\listoffigures
\listoftables

\tcblistof[\section*]{box}{プログラムリスト}

\printindex
\end{document}